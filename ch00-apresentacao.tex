\chapter*{Introduction}
\addcontentsline{toc}{chapter}{Introdução}

% Inspirada no controverso tema da aquisição de verbos irregulares na língua inglesa
% (\cite{Pinker:1999}, \cite{chomsky:1968},  \cite{Pinker:1988}, \cite{rumelhart:1986}), esta pesquisa tem como objetivo estudar o processo de flexão de verbos irregulares do Português Brasileiro fazendo uso de um modelo computacional conhecido como \textit{Encoder-Decoder} (\cite{enc-dec:2014}).

% A controvérsia em torno do processo de aprendizado de verbos irregulares teve início na década de 60, tendo sido explorada por pesquisadores de diversas áreas (linguistas, psicólogos, neurocientistas, cientistas da computação, entre outros). No Cap. \ref{ch:01} deste trabalho, serão apresentadas as origens e motivações da discussão, bem como os principais pesquisadores envolvidos e hipóteses levantadas. Além disso, será introduzida uma categoria de modelos computacionais associativos conhecida como Rede Neural Artificial, do qual o modelo \textit{Encoder-Decoder} faz parte. Em seguida, na primeira parte da seção de motivação (Seç. \ref{sec:motivation}) serão apontadas algumas especificidades gramaticais da língua portuguesa que podem dificultar o processo de aprendizado da flexão irregular. Na segunda parte da seção, serão apresentados alguns trabalhos computacionais que se seguiram nos anos subsequentes em resposta à discussão estabelecida. 

Inspired by the great controversy regarding the acquisition of irregular verbs in the English language
(\cite{Pinker:1999}, \cite{chomsky:1968}, \cite{Pinker:1988}, \cite{rumelhart:1986}), this research aims to study the inflection process of Brazilian Portuguese irregular verbs making use of a computational model known as \textit{Encoder-Decoder} (\cite{enc-dec:2014}).

The debate surrounding the process of learning irregular verbs started in the 1960s, having been explored by researchers from different fields (linguists, psychologists, neuroscientists, computer scientists, among others). In Chap. \Ref{ch:01}, we present the origins and motivations of the discussion, as well as the main researchers involved and raised hypotheses. In addition, a category of associative computational models known as the Artificial Neural Network will be introduced, of which the \textit{Encoder-Decoder} model is part. Next, in the first part of the motivation section (Sec. \Ref{sec:motivation}) we point out some grammatical specificities of the Portuguese language that could hinder the learning process of irregular inflection. In the second part of the section, we will present a historical review regarding the computational contributions that followed in response to the discussion.

% Após a exposição dos diferentes trabalhos desenvolvidos, ficará evidente a razão pela qual tal tema chamou a atenção de tantas áreas distintas. Ademais, veremos que o problema da flexão de verbos irregulares tornou-se um desafio computacional para além dos problemas linguísticos ou cognitivos em questão, de modo que muitas pesquisas subsequentes acabaram se distanciando do debate linguístico e focando nos aspectos matemáticos que viabilizariam o aprendizado artificial. A presente pesquisa também dará continuidade a esse aspecto computacional do problema e não terá como objetivo assumir qualquer posição em torno dos aspectos cognitivos do debate. Desse modo, avaliaremos o uso do modelo \textit{Encoder-Decoder} a partir de um ponto de vista prático de acordo com a sua performance na tarefa proposta e em comparação a outros modelos computacionais já apresentados em outras pesquisas.

After the exhibition of the different conducted studies, it will become evident why this theme caught the attention of so many different areas. Moreover, we will see that the problem of inflecting irregular verbs has become a computational challenge beyond the linguistic or cognitive issues under debate, in such a way that many subsequent researches have ended up distancing themselves from the linguistic matters and focused on the mathematical aspects that would make artificial learning feasible. This research will also continue this computational aspect of the problem and will not aim to take any side regarding the cognitive aspects of the debate. In this way, we will evaluate the use of the \textit{Encoder-Decoder} model from a practical point of view according to its performance in the proposed task and in comparison to other computational models already presented in other research.

To achieve the proposed objective, the first stage of this research was the development of a specific linguistic corpus for the task. The resulting corpus is presented in full in the Appendix (B), but will be discussed in detail in Chap. 2, in Sec. \ref{sec:corpus}. Still in this chapter, we will see the importance of applying appropriate preprocessing treatments in the units of the corpus to enable the intended learning. For that, we will revisit a preprocessing algorithm proposed in a previous work, carried out by \cite{rumelhart:1986}. Then, we will present the preprocessing algorithm developed in this research.

Chapter 3 introduces basic concepts regarding the subject of \textit{Machine Learning} and then present an introduction to Artificial Neural Network models. In addition, we will address the concepts of \textit{Language Models} and architectures of \textit{Recurring Neural Networks} - essential issues for understanding the highlighted model of this research, the \textit{Encoder-Decoder}. After making the necessary introductions, we will be ready to present the \textit{Encoder-Decoder} model in Chapter 4.  

Chapter 5, in turn, will present and discuss the results obtained. In it we will see the settings chosen for the definition of the model and the metrics used for its evaluation. We will also analyze the different errors observed and look for possible explanations for the results obtained.

To conclude, Chapter 6 will display a summary of the subjects covered in this research and also point out suggestions for future research on the subject.




